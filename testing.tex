\section{Testing (\textit{Jacob Salomonsen})}
The object of the testing is to find if an unoptimized GPU version of the D2Q9 LBM will outperform the supplied CPU, in terms of running time.

\subsection{System description}
The tests were run on a MacBook Pro with specs listed in \hpref{table}{specs}

\begin{table}[htb]
	\centering
	\begin{tabular}{ll}
		\toprule
		Spec & Value \\
		\midrule
		Processor & 2.66 GHz Intel Core i7 \\
		Memory & 4 GB 1067 MHz DDR3 \\
		GPU & NVIDIA GeForce GT 330M on PCIe x16\\
		VRAM & 512 MB\\
		CUDA version & 3.20\\
		Compute capability & 1.2\\
		Compute cores & 48\\		
		\bottomrule
	\end{tabular}
	\caption{Specs of the system used for testing}
	\label{specs}
\end{table}

\subsection{Test design}
Testing is performed with the same settings for the two versions (CPU and GPU). This meaning that the amount of iterations made for the simulation is the same. The obstacle scenario is also the same, since this might affect the simulation considering some of the operations are different for nodes which are obstacles. 

The parameter which is changed is the simulated field. Here the parameter is tested with an increment of 16, where the field is always quadratic. The testing for each field size is performed in a sequence of three, and the average of these three results is taken. This is done both for the CPU- and the GPU-version.

\subsection{Timing}
The timing is performed by Pythons built-in timing class, called \textit{timeit}. This class can time specific parts of a program given that they are defined as functions with the \texttt{def} keyword. Only the loop which iterates over the nodes is included in the function definition so initialization latency is not included in the calculation of running time.

\subsection{Scenarios}
To visually compare with others implementation of the LBM a special test case has been devised, called \emph{the lid driven cavity}. This is a special setup which can help the implementer identify if the simulation behaves according to a standardized and accepted behaviour. This will not form the basis for any conclusions about the correctness of the LBM simulation as a whole keeping in mind this is only a black-box implementation. It is though interesting to see if the simulation performs as one would expect.