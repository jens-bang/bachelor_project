%%%--- TITEL OG FORFATTER ---%%%
\newcommand{\tit}{Benchmarking the Lattice Boltzmann model implemented using NumPy and PyCUDA}
\newcommand{\aut}{Jens Bang \& Jacob Salomonsen}
\newcommand{\subj}{Bachelor Project}
\newcommand{\mail}{\href{mailto:bachelor@snej.dk}{bachelor@snej.dk} or \href{mailto:jiekebo@gmail.com}{jiekebo@gmail.com}}

\newcommand{\subfigureautorefname}{\figureautorefname}

%%%--- PREAMBLE INCLUDE ---%%%
\input{preamble.tex}
\usepackage[numbered,framed]{mcode}
%\usepackage[danish]{babel}				% Dansk orddeling osv.
%\usepackage[fixlanguage]{babelbib}		% Sprogpakke til BibTeX
%\selectbiblanguage{danish}				% Sprogvalg til BibTeX

%%%--- LISTISTINGS SETUP ---%%%
\lstset{
language=Python, 						% choose the language of the code
basicstyle=\scriptsize,					% the size of the fonts that are used for the code
%numbers=left,							% where to put the line-numbers
%numberstyle=\footnotesize,				% the size of the fonts that are used for the line-numbers
%stepnumber=2,							% the step between two line-numbers. If it's 1 each line will be numbered
%numbersep=5pt,							% how far the line-numbers are from the code
%backgroundcolor=\color{white},			% choose the background color. You must add \usepackage{color}
%showspaces=false,						% show spaces adding particular underscores
%showstringspaces=false,					% underline spaces within strings
%showtabs=false,							% show tabs within strings adding particular underscores
%frame=single,							% adds a frame around the code
%tabsize=2,								% sets default tabsize to 2 spaces
captionpos=b,							% sets the caption-position to bottom
%breaklines=true,						% sets automatic line breaking
%breakatwhitespace=false,				% sets if automatic breaks should only happen at whitespace
%escapeinside={\%*}{*)}					% if you want to add a comment within your code
}

%%%--- FANCYHDR ---%%%
\pagestyle{fancy}
\chead{}
\lhead{\aut}
\rhead{\nouppercase{\leftmark}}
\cfoot{}
\lfoot{\today}
\rfoot{\thepage}

%%%--- DOKUMENT STARTER HER ---%%%
\begin{document}

%%%--- Title ---%%%%
\begin{titlepage}
\HRule
\begin{center}\huge{\bfseries{\tit}}\end{center}
\HRule
\\[1.0cm]
\begin{center}
\aut
\\[0.5cm]
\mail
\\[0.5cm]
Supervisor - Professor Brian Vinter
\\[0.5cm]
\subj
\\[1.5cm]
Department of Computer Science
\\[0.5cm]
University of Copenhagen
\end{center}
\vfill
\begin{center}\today\end{center}
\end{titlepage}

%%%--- Tekst ---%%%

\section*{Abstract (\textit{Jens Bang})}

The Lattice Boltzmann Model is widely used for various simulations of movement of liquids and gases. This paper will show that it is possible to achieve even high gains in execution speed by combining ease of implementation and parallel processing, by implementing the Lattice Boltzmann Model in Python, using the PyCUDA extension to access the highly parallel CUDA architecture. We will compare run times between the PyCUDA version and a version utilizing the more conventional, and not parallel, NumPy extension, to show the speed increases.

\newpage
\tableofcontents
\newpage

\section{Introduction (\textit{Jens Bang})}

This paper is a bachelor thesis written by bachelor degree students at The Computer Science Department of The University of Copenhagen. As such the intended readers are anyone with an interest in the subject matter, as well as a basic understanding of computer science and programming in Python. The object of the thesis is to examine a possible gain in execution speed, when using CUDA to implement the Lattice Boltzmann Model. It is not the object of the thesis to fully explain the Lattice Boltzmann Model, so while a reasonable understanding of physics is preferrable, it is not necessary.

When we started working on the project, the plan was to implement the Lattice Boltzmann Model in two version, both using CUDA. One version in Python, and one version in C/C++. The object was to compare CUDA implementations in Python and C/C++, to see if the ease of implementation in Python would carry with it a loss of execution speed, as compared to the execution speed of the more cumbersome implementation time of C/C++.

During the initial work on the project we realized that writing CUDA code in Python is done by simply embedding CUDA C code in the Python code. This CUDA C code will then be passed on to precisely the same compiler as the CUDA code in the C/C++ version. This would essentially mean that comparing the run times of the two implementations, would only reveal the differences between the run times of the initializing code, and would reveal nothing about the run times of the CUDA itself.

Since this would be a comparison of normal C/C++ code run times to normal Python code run times, with no CUDA involved, a case that has already been extensively documented, and since we wanted to investigate a possible increase/decrease in execution speed when implementing CUDA code in Python, as opposed to implementing the same CUDA code in C/C++, going further would clearly not accomplish anything new. For this reason we, in agreement with out advisor, changed the scope of the thesis to investigate possible speed increases or decreses, when implementing the Lattice Boltzmann Model in Python using PyCUDA, as opposed to implementing the same model in Python using NumPy.

The two Python implementations of the Lattice Boltzmann Model are both based on a MatLab implementation of the model, which will be introduced in the Matlab implementation analysis section (\autoref{sec:matlabimplementation}) of this thesis.

\newpage

\section{Abbreviations}

\begin{figure}[htb]
\centering
	\begin{tabular}{lcl}
	    CPU & = & Central processing unit\\
	    CUDA & = & Compute Unified Device Architecture\\
		D2Q9 & = & Two dimensional, 9 directional vectors per lattice-point\\
		D3Q19 & = & Three dimensional, 19 directional vectors per lattice-point\\
		GPU & = & Graphics processing unit\\
		LBM & = & Lattice Boltzmann Method\\
		PTX & = & Parallel Thread Execution\\
		SIMD & = & Single instruction, multiple data\\
		SIMT & = & Single Instruction Multiple Threads
	\end{tabular}
\end{figure}

\newpage
\section{Theory}

\subsection{Lattice Boltzmann Model (\textit{Jens Bang})}
The Lattice Boltzmann Model (LBM) is a simplified version of the Boltzmann equation, which simulates the behaviour of fluid flows. The simplification consists of limiting the particles to only occupy certain points in space (vertices in a lattice), and to only travel along specified directional vectors, with constant speed. 

In this way the Lattice Boltzmann Model (LBM) describe simple fluids (gas and liquids), i.e. it ignores thermal effects and tracers. The LBM simulates movements in fluids by looking at particles found at points in a lattice at discrete time-steps. There are different versions of the LBM using either 2-dimensional or 3-dimensional lattices.

Each point in the lattice has a set of state-variables attached, describing the state of the particle found at that point. Each point also has a set of fixed directions, along which the particle can travel. In a 2-dimensional lattice there are normally 9 directions, while in a 3-dimensional lattice you see either 15 or 19 directions. These 3 setups are normally referred to as D2Q9, D3Q15 and D3Q19.

\insfig{./images/d2q9_d3q19.png}{0.5\textwidth}{LBM directions (TODO: Insert reference to Scholarpedia)}{lbmdirections}

The LBM divides the flow of gasses or fluids into small, discrete time-steps. For each time-step the algorithm traverses all lattice points and for each point calculates the velocity and direction the particle travels in, as well as handling any collisions between particles.

\subsubsection{Using the LBM}
Since the LBM is especially well suited to simulate flows around even very complex geometric structures, it has a wide variety of practical uses. Among these are:
\begin{itemize}
\item Simulating pore-scale processes in porous media [TODO Citation: BrittSBChristensenPhD-Thesis.pdf]
\item Wind turbines [TODO Citation needed]
\item Bridge pillars [TODO Citation needed]
\item Others?
\end{itemize}

\subsection{CPU vs. GPU}
  Different goals produce different designs !   GPU assumes work load is highly parallel !   CPU must be good at everything, parallel or not
!   CPU: minimize latency experienced by 1 thread !   big on-chip caches !   sophisticated control logic
!   GPU: maximize throughput of all threads
!  
!   !  
threads in flight limited by resources => lots of resources (registers, bandwidth, etc.)
multithreading can hide latency => skip the big caches share control logic across many threads

\subsection{Parallel processing}

\subsection{CUDA (\textit{Jacob Salomonsen})}
In the beginning, GPU's had only very limited purposes. Mostly generating real-time graphics for games, but also in a few cases for production and scientific applications. This has changed since the proliferation of the pixel shader. A pixel shader basically produces a color for every point on a screen, by taking into account the (x,y) position of the pixel, the light settings of a scene, the material properties of objects in the scene and so on. Early researchers found that this ability to perform computations on each pixel could be harnessed to other uses. Since pixel shaders are completely controlled by the programmer, the possibility of simply giving a GPU data as input, instead of a scene to render, was within reach. This would mark the beginning of the general purpose GPU.

The general purpose GPU did however lack a platform for developers to build upon, since learning to program pixel shaders required previous knowledge about either openGL or DirectX. Also even when knowing these frameworks, the model would lack the perspective of general purpose computing and instead be focused on generating graphics, where the general purpose computing aspect would be achieved by "cheating" the GPU into treating data as if it was a scene to be rasterized.

Enter CUDA. CUDA is nVidia's way of introducing general purpose GPU programming to a wider audience. CUDA utilizes the parallel nature of the GPU, allowing a low end computer to process data in a different, and some times more efficient, way.

The CUDA API exposes a set of tags, which allows a programmer to easily specify what methods, in otherwise ordinary C-code, should be executed on the GPU. The way CUDA makes it possible is by using a preprocessor which parses the source code and makes different files for the individual architectures. Thus the normal C-code goes to whichever compiler the user chooses, and the parallel section of the code goes to a compiler specifically designed to generate PTX-code. PTX-code is essentially assembly which is not specific to any GPU. The PTX code is then translated into assembly code specifically targeted at the GPU.

This effectively separates the tedium of writing assembly code directly aimed at a specific GPU-architecture, or writing pixel shaders via graphics APIs.

\subsubsection{CUDA programming model}
The CUDA programming model efficiently supports the steps needed to be taken, in order to make a process parallel. The model is arranged so a decomposition of the problem area is naturally ingrained in the programming procedure, by having the programmer split the problem into parts. Since the GPU is capable of launching many threads, it is necessary to have control of what thread is launched where. Therefore nVidia has structured execution of code in a way, that both allows an easy intuition of the GPU-architecture, as well as supporting the steps in making a process parallel.

The code to be executed on the GPU, \emph{the kernel}, launches a grid which contains blocks of threads. A kernel can only launch one grid at a time. Blocks can be arranged in two dimensions, and the threads within the blocks can be arranged in three dimensions. Threads within blocks can cooperate via shared memory, but cannot cooperate with threads from another block. It is therefore pivotal for the performance of a parallel process that a domain analysis has been performed, and the memory intensive parts of the problem are contained within one block.

Blocks do not only have the role of sharing memory among threads. They also serve as a synchronization point for the threads within themselves. A special command within CUDA will make sure that all threads within a block have completed, before further execution is permitted.

To support this structure, CUDA provides built-in facilities for indexing blocks and threads. This allows the programmer to write code specifically for a single thread within a grid of many blocks, together containing millions of threads.

(TODO: make figure to show how grid and block structure can be in cuda)

\insfig{./images/cudaprocess.png}{0.5\textwidth}{CUDA process model (from Wikipedia article, author: Tosaka)}{cudaprocess}

\subsubsection{CUDA data model}

\insfig{./images/cudamem.png}{0.5\textwidth}{CUDA(GPU) memory model (TODO: where did I get this from?!)}{cudamem}


\newpage
\section{Implementation}

\subsection{Assumptions}

\subsection{Problem domain}

\subsection{Code Analysis}
The initial code analysis was done to get an overview of the code at hand.

\subsection{Design choices}
In this section I will describe the choices made during the implementation of the Lattice Boltzmann Model. The implementation takes origin in the Matlab code included in \hpref{listing}{lbmmatlab}.

To be able to perform a proper comparison between the CPU- and GPU-version of the model, a NumPy version has been created also taking origin in the Matlab code. This allows for the same benchmarking technique within the Python code, using the (TODO: insert library name here).

\subsubsection{PyCUDA}
PyCUDA is an extension for the Python language, that allows for full access to functionality available to CUDA C. Python is a scripting language, which means it is not compiled but interpreted. This leverages the programming process, and allows for some of the advantages of an interpreted language. One notable advantage over CUDA C is automatic resource control. Another advantage is the tight coupling with NumPy, which is especially good for the main purpose of this report. The testing procedure will gain more reliability by keeping the execution on the same platform.

PyCUDA allows a programmer trained in the usage of CUDA C, to easily start programming. Basically PyCUDA allows for CUDA C-code to be entered directly into the Python script. This is afforded by the \texttt{SourceModule}, which is a command used for creating CUDA-kernels. The \texttt{SourceModule} utilizes a just-in-time compilation process. (TODO: check fact!) This means that the kernel is only compiled at the moment it is needed.

As with CUDA C it is also necessary to load data onto the GPU by first allocating space and then copying to the GPU-memory. PyCUDA allows for allocating memory on the GPU using the \texttt{driver.mem\_ alloc} command, with argument for the size of the memory to allocate. The copying is performed with the \texttt{device.memcpy\_ htod}. Copying back from the GPU is afforded by the command \texttt{driver.memcpy\_ dtoh}. These commands allocate global memory on the GPU by default. (TODO: check whether the copy is just for a pointer to the python buffer or if data is loaded into gpu-memory)

\subsubsection{NumPy}

\subsubsection{Data structure}
The D2Q9 LBM maintains velocities for eight directional states plus one stationary state per point in the simulated field. If the dimension of the simulated field is $x \cdot y$ then to be able to contain this, the data structure must be able to contain $9 \cdot x \cdot y$. blabla

\newpage
\section{Testing}

\subsection{System description}

\subsection{Test design}



\newpage
\section{Results (\textit{Jacob Salomonsen})}

\insfig{./images/result.png}{1.0\textwidth}{results}{results}

\insfig{./images/lidref.png}{1.0\textwidth}{lidref}{lidref}

\insfig{./images/cuda_lid_32_32_900.png}{1.0\textwidth}{lidref}{lidref}

\insfig{./images/cuda_lid_32_32_900_color.png}{1.0\textwidth}{lidref}{lidref}

\insfig{./images/cuda_lid_256_256_2700.png}{1.0\textwidth}{lidref}{lidref}

\insfig{./images/cuda_tunnel_32_32_900.png}{1.0\textwidth}{lidref}{lidref}

\insfig{./images/cuda_tunnel_32_32_900_color.png}{1.0\textwidth}{lidref}{lidref}

\newpage
\section{Conclusion}

The revised goals of this thesis was to show the increase in execution speed, when switching a Python implementation of the Lattice Boltzmann Model from NumPy to PyCUDA. While we realize that comparing the execution speed of a NumPy, which always only runs on one core, and PyCUDA, which is highly parallel in nature, running on multiple cores, scientifically speaking is like comparing apples and oranges, we wanted to show the gain in a real-life setting, and so we feel that the comparison is valid.

As outlined in \autoref{usingthelbm} the Lattice Boltzmann Model is widely used in many different circumstances, to model a large variety of things. This is why we feel it is important to show the speed increase, when switching from NumPy to PyCUDA, because that is the benefit any programmer will see, when implementing the same switch from NumPy to PyCUDA.

We started by writing a Python implementation of the supplied Matlab script, using NumPy. When we were certain of the correctness of this implementation, we wrote the PyCUDA version of the Python script, and tested the correctness of this new implementation. Then we performed time studies to see what kind, if any, of speed increase we would get. As expected the speed increase is quite significant, especially with larger problem spaces.

As mentioned in \autoref{sec:runningtime} the graph shapes in \autoref{results} are reminiscent of quadratic functions. But while the execution time the NumPy version very quickly becomes unmanageable, the execution time for the PyCUDA version increases at a much slower rate. In fact the PyCUDA version is able to run problem spaces almost 3.5 times larger than the NumPy version in the same time frame.

We have clearly done what we set out to do, which was to show a speed increase when switching a Python implementation from using NumPy to PyCUDA.

\subsection{Where to go from here?}

There are two obvious ways to go from here:

\begin{enumerate}
\item As mentioned in \autoref{sec:testing} the PyCUDA version is unoptimized, and it could be very interesting to see what kind of speed gains there is to be found by optimizing the code.
\item Another interesting angle would be to examine if OpenCL is comparable to CUDA, or if it delivers even greater speed increases.
\end{enumerate}

\newpage
\section{Appendix}

\lstset{basicstyle=\scriptsize, caption={Lattice Boltzmann Model, implemented in Matlab},label=lbmmatlab}
\lstinputlisting{./code/lbm2d.m}

\newpage
\section{Bibliography}

To ensure access to the various on-line documents we have used as references, we have created a page with PDF files produced from the documents that were available when we published our thesis: \url{http://www.snej.dk/lbm/}.

\bibliographystyle{plain}
\bibliography{library}
\addcontentsline{toc}{section}{References}

%%%---BibTeX ---%%%
%(Compile: 1 x LaTeX 1 x Bibtex 2 x LaTex)%

\end{document}
